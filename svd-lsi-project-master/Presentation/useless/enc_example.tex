\begin{frame}[plain]
	\frametitle{[Curran02] More concrete example}
	\begin{block}{}
    A context relation is a tuple (w, r, w')
    where w is a thesaurus term, occurring in relation
    type r, with another word w' in the sentence. The
    type can be grammatical or the position of w' in a
    context window: (dog, direct-obj, walk). If tuple (r, w') is
    treated a single unit (attribute of w), it fits in a matrix.
	\end{block} 
	\begin{block}{}
    The simplest method implemented extracts the
    occurrence counts of words within a particular window surrounding
    the thesaurus term (saved in $v[i]$, where vector $v$ represents
    the word).
	\end{block} 
	\begin{block}{}
    Morphological analysis (OLT) on of both the words
    and attributes, reduces the
    representation space and number of unique 
    context relations (eg coalescing the plural nouns with their
    corresponding singular nouns); which also reduces data sparseness
    problems). 
	\end{block} 
\end{frame}
