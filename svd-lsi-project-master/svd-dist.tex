\chapter{Distributed SVD algorithm}
\label{cha:svd-dist}

In the previous chapter we discussed the state of the art, regarding
the serial version of the SVD algorithm, on the context of the LSI
problem; such algorithm was discussed under the assumption that
the matrix and auxiliary artifacts fit into the RAM of such
computer. We also mentioned that such algorithm, could benefit from
parallel or vectorized linear algebra kernels; speciall for its most
expensive operations (like the sparse matrix-vector
multiplication). In this chapter, we will discuss a chosen distributed version
of SVD algorithm; where the calculation is spread across computing
nodes in a cluster, aiming mainly to scale in time (due the inherent
parallelization). \\

The most scalable and documented algorithm for SVD-LSI, that we found
in literature, was that of Radim \Rehurek; who published his results
into a series of articles (\cite{rehurek10a},
\cite{rehurek10b} and \cite{rehurek11b}), and culminated the effort
with his PhD thesis (\cite{rehurek11a}). All the articles are
  pretty much contained  in \Rehurek's Phd thesis, then unless stated
  explicitly, all the references to his work in this chapter will be
  from that publication. Is is 
fair to emphasize though, that \Rehurek thesis covers other topics
besides those we care about in this project; thus, we focused on his
chapter of SVD/LSI only. \\

Another pertinent clarification about \Rehurek's work, is that he
offers the distributed algorithm mostly as a way of speeding up the
SVD calculation (scale in time); and not precisely for tackling bigger
problems that simply do not fit in the memory of a single computer
(scale in space). On the large scale experiments that he reports, the
resulting matrix can pretty much fit into the RAM of a modern personal computer;
actually, he uses that to compare the reduction in time on the serial
execution (single machine) vs the distributed execution (cluster).

\section{The one-pass distributed algorithm}

The essence of the distributed strategy is to achieve almost perfect
parallelism, by splitting the input matrix into several smaller
matrices called \emph{jobs}. \\

\[
A^{m \times n} = 
\begin{bmatrix}
A_1^{m \times c_1} \mid A_2^{m \times c_2} \mid \cdots \mid A_k^{m \times c_k}
\end{bmatrix}
\suchthat \sum_{i=1}^k c_i = n
\]
\\

A subset of these smaller matrices or \emph{jobs} is assigned to each
node in the cluster, depending on their capabilities; the
objective is to assign matrices that fit into the node's RAM
memory. Each node will calculate the SVD factorization of the
submatrices assigned, but merging those results into a single
SVD approximation that covers all the input data it received. At the
end, a global merge step across all the nodes is performed, giving the
global SVD approximation for original matrix $A$. The
\cref{alg:svd-dist} describes the overall distributed algorithm: \\

\begin{algorithm}
  \label{alg:svd-dist}
  \caption{Distributed-SVD: Distributed SVD for LSI (global)}
%
  \setstretch{1.35}
  \SetKwInOut{Input}{Input}
  \SetKwInOut{Output}{Output}
  \DontPrintSemicolon
%
    \Input{Truncation factor $k$, queue of jobs $A= [A_1, A_2, \dots ]$}
%
    \Output{Matrices $U^{m \times k}$ and $\Sigma^{k \times k}$, 
      from the SVD decomp. of $A$}
%
    \For {\textbf{all} (node $i$ in cluster)}
    {
      $B_i \gets \text{subset of the queue of jobs } [A_1,A_2,\dots]$ \;
%
      $P_i = (U_i,\Sigma_i) \gets \func{SVD-Node}(k,B_i)$ \;
    }
    $(U,\Sigma) \gets \func{Reduce}(\func{Merge-SVD},[P_1,P_2,\dots])$ \;
%
    return $(U, \Sigma)$ \;
\end{algorithm}
\hfill

The first important detail from the algorithm just shown, is that we
are not calculating the matrix $V$ from the SVD factorization, how
come! Such detail is explained at the end of the last section. For the
moment, let us just say that such matrix is not required for our
purposes. \\

We can also observe the map-reduce pattern in this algorithm, with the map
part being the iteration done over $p$ nodes (in parallel); and the
reduce part being the final merge of those partial results. The
\cref{alg:svd-dist-node} describes the part done inside each node.

\begin{algorithm}
  \label{alg:svd-dist-node}
  \caption{SVD-Node: Distributed SVD for LSI (node)}
%
  \setstretch{1.35}
  \SetKwInOut{Input}{Input}
  \SetKwInOut{Output}{Output}
  \DontPrintSemicolon
%
  \Input{Truncation factor $k$, queue of jobs $A_1,A_2,\dots$}
%
  \Output{Matrices $U^{m \times k}$ and $\Sigma^{k
        \times k}$, from the SVD  of $[A_1,A_2,\dots]$}
%
  $P = (U,\Sigma) \gets 0^{m \times k} 0^{k \times k}$ \;
%
  \For {each job $A_i$}
  {
    $\prim{P} = (\prim{U},\prim{\Sigma}) \gets \func{Basecase-SVD}(k,A_i)$ \;
%
    $P = (U^{m \times k},\Sigma^{k \times k}) \gets \func{Merge-SVD}(k, P, \prim{P})$ \;
  }
%
  return $(U,\Sigma)$ \;
\end{algorithm}
\hfill

It is important to realize that the iteration in this
\cref{alg:svd-dist-node} is done serially, but that the procedure
$\func{Basecase-SVD}$ that resolves the SVD of a
matrix that fits in memory (base case), internally may exploit the
multicore or vectorial capabilities of the node computer. This
procedure serves as a black box SVD calculator, and \Rehurek mentions
at least two algorithms which can be plugged on its place: \\

\begin{enumerate}
\item The Lanczos algorithm as implemented by SVDLIBC (\cite{svdlibc}),
  which in turn is based on SVDPACKC written by Berry et al
  (\cite{svdpackc}), which in turn is based on its Fortran77
  predecessor SVDPACK (\cite{svdpack}). All of them  ultimately based
  on seminal paper by  Berry \cite{berry92} (which in turn comes from
  his PhD thesis \cite{berry91}). \\
\item A custom stochastic algorithm based on the work of Halko et al
  (see \cite{halko11}).
\end{enumerate}
\hfill

For the scope of this project, we considered appropriate to focus only
on the Lanczos based algorithm; as that is essentially what we
described in the previous chapter. In that sense, the work of \Rehurek
is interesting because by using the divide and conquer strategy for
the SVD problem, he is leveraging on the decades of research and
numerical accuracy of the work done by Berry et al. At the same time,
his key contribution becomes the procedure $\func{Merge-SVD}$, which
we will describe in further sections. \\


\section{Subspace tracking}

\Rehurek does a very comprehensive survey of the state of the art
regarding SVD algorithms, in order to position the variant that he
proposes in a wider context. This is because there are a lot of
variants of SVD algorithms over there, each one emphasizing a
different subset of aspects. Actually, let us remember that we have
already restricted ourselves in a couple of aspects, when we focused
our attention to the particular application of LSI: \\

\begin{itemize}
\item The LSI term-document matrices are highly sparse, which allows
  one to prune a big branch of the SVD tree of algorithms. 
\item The LSI applications require a truncated SVD factorization,
  usually of a few hundred entries; therefore, algorithms that take
  advantage of such truncation are preferred. 
\end{itemize}
\hfill

\Rehurek goes even further on this specialization approach, and
imposes himself additional restrictions:

\begin{enumerate}
\item Distributable: he seems specially interested in achieving a
  high-level parallelization of the problem, that can be split across
  the nodes of a commodity cluster.\\

\item Online: contrary to a batch SVD algorithm, he is interested in
  an algorithm that is capable of 
  reusing the already computed SVD factorization, in such a way that
  one can update previous solution when new data comes available. This
  can be useful in today's applications for LSI, which may get the
  documents from social networks or similar environments that can be
  thought as a permanent and basically unlimited source of
  data. Recalculating from scratch the SVD may be unfeasible under
  those circumstances, hence updating an existing solution is
  desired.\\

\item One pass: as discussed in previous chapter, among the most advanced
  parallel algorithms for SVD use the so called approach of Krylov
  subspaces (Lanczos,Arnoldi); they do require though, several passes
  to the data. But \Rehurek is interested on streamed environments,
  where saving all the data may be just unfeasible; hence, he proposes
  instead an algorithm that consumes the data in a single pass and
  discards it. This applies again to the documents of the LSI 
  problem; let us recall that the term-matrix of $m \times n$ is very
  wide in the horizontal sense ($n >> m$); this situation comes from
  the fact that we have much more documents (columns) than terms
  (rows). Putting again the sample application that extracts the
  documents from social networks, the terms used for English
  language is typically around $100,000$ and is assumed to be static, while
  documents can be generated constantly in volumes of
  millions. Wanting to accumulate them all, for the sake of having the SVD
  factorization that covers everything, does not seem practical
  either; hence, on a given time we update existing solution with new
  data and immediately discard it (keeping only results the
  factorization). \\

\item Constant memory: strong emphasis is placed on the memory
  complexity of the algorithm, aiming to avoid dependency on the input
  data. The memory complexity of an algorithm that saved all
  documents, historically, can be seen just as \bigO{n}, where $n$
  is the number of columns of the term-document matrix. But the
  distributed algorithm ensures that memory requirements are
  controlled, and depend mostly on the size of truncated matrix (which
  is usually a few hundreds for LSI).
\end{enumerate}
\hfill

An algorithm that posses all these attributes: online, one pass and
using constant memory, can be considered an instance of the so called
``subspace tracking'' approach. The term may not intuitively reflect
all the properties, but we tried to come up a justification of the
name. It may come from the fact that the
SVD factorization, among several other factorizations, essentially
gives us subspaces that characterize our input matrix (recall that the
matrices $V$ and $U$ contain basis for the four subspaces
$\C{\trans{A}}, \N{A}, \C{A},  \N{\trans{A}}$). As we
update our factorization due new data, such subspaces may change; then, by
continuously updating the basis (SVD matrices) we could say that the
subspace they generate is being ``tracked'' across
time \footnote{This attempt to explain the origin of the 
  term is of our own, and is just to help one understanding the term of
  ``subspace tracking''; but texts and books about it, usually in the
  area of Signal Processing, do not seem to explain the concept in this way.}. 

It may not be evident but the characteristics imposed on the algorithm
for being one-pass and online, actually imply that we can not store
the matrix $V$ from the SVD factorization. As suggested above, such
storage is prohibitive because its dimensions $n \times n$, which come
directly from the number of documents to handle throughout time (which
taken historically, can be a huge amount). An essential variant of the
algorithm proposed by \Rehurek then, is that it just deals with the
calculation of the matrices $U$ and $\Sigma$; leaving $V$ behind. How
is that possible? Please refer to section \cref{sec:svd-lanczos-eigen}
for an explanation of the equivalence between the SVD problem, and the
Eigenproblem of either symmetric matrix $\trans{A}A$ or $A\trans{A}$. \\

Therefore, if we are just interested in matrices $U$ and $\Sigma$ of
the SVD factorization; we can restate our goal as solving the
eigenproblem for symmetric matrix $A\trans{A}$. That is, finding its
eigenvalues ($\Sigma^2$) and eigenvectors ($U$). \\

Before proceeding to review the details of procedure
$\func{Merge-SVD}$, which serves to merge two SVD factorizations, is
important to clarify that \Rehurek uses the matrix $P$ in his
pseudocode as a tuple $(U,\Sigma)$ rather than as the product
$\inv{\Sigma}\trans{U}$. This is due practical reasons, as we need to
individually access the original matrices; but still the name $P$ is
kept, to remind us that they can form the projection matrix. 

\section{Merging Two SVD factorizations}

The core logic of the algorithms presented in last section
(\cref{alg:svd-dist} and \cref{alg:svd-dist-node}), relies on the
procedure \func{Merge-SVD}. Is may not be evident at all, but the
essence of this merge is to use SVD factorization again! The The PhD
thesis of \Rehurek presents a series of refinements, until he reaches
the optimized version presented below: \\

\begin{algorithm}
  \label{alg:merge-svd}
  \caption{$\func{Merge-SVD}$: Merge of two SVD factorizations}
%
  \setstretch{1.35}
  \DontPrintSemicolon
  \SetKwInOut{Input}{Input}
  \SetKwInOut{Output}{Output}
%
  \Input{Truncation factor $k$, decay factor $\gamma$,  
    $P_1 = (U_1^{m \times k_1}, \Sigma_1^{k_1 \times k_1})$,
    $P_2 = (U_2^{m \times k_2}, \Sigma_1^{k_2 \times k_2})$}
%
  \Output{$(U^{m \times k}, \Sigma^{k \times k})$}
%
  $Z^{k_1 \times k_2} \gets \trans{U_1}U_2$ \;
%
  $\prim{U} R \xleftarrow{QR} U_2 - U_1 Z$ \;
%
  $U_R \Sigma\trans{V_R} \xleftarrow{SVD_k}
    \begin{bmatrix}
      \gamma\Sigma_1 & Z \Sigma_2 \\
      0 & R\Sigma_2
    \end{bmatrix}^{(k_1 + k_2) \times (k_1 + k_2)}$ \;
%
  $\begin{bmatrix}
      R_1^{k_1 \times k} \\
      R_2^{k2 \times k}
    \end{bmatrix} = U_R$ \;
%
  $U \gets U_1R_1 + \prim{U}R_2$ \;
%
  return $(U,\Sigma)$ \;
\end{algorithm}
\hfill

The \cref{alg:merge-svd} is a quite compressed piece of work, and none
of its steps are intuitive. We proceed to explain them in more detail
in the following subsections.

\subsection{Input and Output Parameters}

Is worth to remark a couple of new features that appear as input
parameters of the merge procedure: we are introducing a new decay
factor $\gamma \in (0.0,1.0)$ that helps to give less relevance to old
documents. Let us recall that these algorithms are designed to update
an existing SVD calculation, where each update processes a new set
documents (encoded as columns of the term-document matrix $A$). \\

There are three truncation parameters ($k$, $k_1$ and $k_2$), instead of
just one; this is to give further flexibility to the algorithm, as it
supports that the truncation factor varies with time. Each of the
previous factorizations then could have been done with different
truncation factors; but we homogenize the final result with the new
truncation factor $k$. This feature may not be heavily used, as
usually $k$ is fixed in a few hundreds and not changed during the
entire life of the LSI applications; there is no need though, to loose
generality and impose the artificial restriction that the truncation
factor shall remain static. \\

The output parameter, or result of the merge algorithm, is a new
factorization $U,\Sigma$ which covers the two partial SVD
factorizations received. 

\subsection{Construction of a new basis}

Most of the algorithm is about building a new basis (columns of matrix
$U$), that spans the subspaces generated by basis in $U_1$ and $U_2$,
respectively. This is done by taking advantage that $U_1$ and $U_2$
have orthonormal basis already as columns; hence, one of them is
picked ($U_1$), and we only build the delta $\prim{U}$ required to
extend  basis $U_1$ into required basis $U$. \\

The first two lines of \cref{alg:merge-svd} are basically to build the
delta basis $\prim{U}$, and thought not evident (nor explained in the
articles by \Rehurek), we can find an intuitive interpretation of this
procedure. Let us think in two vectors in \R{3} named 
\vec{u_1} and \vec{u_2}, which are linearly
independent. Let us suppose that we are given the task of building a
basis of the two-dimensional subspace that those vectors span, with
the additional requirement of making such basis orthonormal. Let us
suppose that we pick \vec{u_1} to be part of the basis, and now we
just need to find an orthogonal vector to $u_1$, in order to complete our
task. It can be proven that if we subtract from 
\vec{u_2} the projection of \vec{u_2} into \vec{u_1}, we get a vector
that is orthogonal to \vec{u_1} (let us name it $\vec{u_3}$):

\[
\vec{u_3} = \vec{u_1} - (\vec{u_2} \cdot \vec{u_1}) \vec{u_1} \suchthat
\vec{v_3} \ds{\perp} \vec{u_1}
\]
\hfill

If we consider the resulting set from the above recipe, that is $\{\vec{u_1},
\vec{u_3}\}$, we can not tell yet that is an orthonormal
basis. However, they are at least linearly independent, hence we can
apply standard procedures like Gram-Schmidt (see \cite{strang88}) to
produce the desired orthonormal basis. \\


Of course the above recipe works for any dimension, and that is
essentially the calculation done in the first two lines of
\cref{alg:merge-svd}; though it 
states all the vector equations at once, by using matrix
notation (the columns of matrices $U_1$ and $U_2$ play the role of
vectors \vec{u_1} and \vec{u_2} from our example; and the right side
of assignment of line two corresponds to vector \vec{u_3}). On the
first line we calculate matrix $Z$ which is the projection matrix of
the columns of $U_2$ into columns from $U_1$; this give us the
component of the projections only (the dot products), but multiplying
that by $U_1$ is equivalent to the expression $(\vec{u_2} \cdot
\vec{u_1}) \vec{u_1}$ from our example. The matrix
substraction is a compressed way of introducing the vector equations
from our example; and the QR factorization used to produce the
orthonormal basis, is basically the application of the Gram-Schmidt
process that we mentioned as well. It is not mentioned by
\cite{rehurek11a} but the way of calculating 
$\prim{U}$ is quite similar (if not the same), to the one reported by
Hall et all in \cite{hall00} and \cite{hall02} (where it is done in
the context of merging eigen models, which in particular contain eigen
decompositions). \\

The usage of factorization $QR$ deserves more comments, as we have not
mentioned much about it until now. Given a rectangular matrix $B^{m
  \times n}$, it produces a factorization which consists of an
  orthogonal matrix $Q^{m \times m}$ (which is essentially a basis
  for the subspace spanned by the columns of $A$); followed by an
  upper triangular matrix $R^{m \times 
    n}$. The triangular form of $R$ 
  comes from the application of the Gram-Schmidt algorithm: the column
  $R_1$ contains the coordinates of original column $A_1$ respect to
  the basis $Q$ (it only depends on $Q_1$), the column $R_2$ indicates
  that original column $A_2$ depends only on the first two columns of
  $Q$, and so on. The QR algorithm is chosen by \Rehurek, not only due
  its ability to produce the missing vectors we needed for our basis
  (matrix $\prim{U}$); but also due its side product, the triangular matrix
  $R$ which is used in further steps.

\subsection{Producing the diagonal matrix $\Sigma$}

The probably most obscure step appears in line $3$, where another SVD
factorization is being applied, in order to produce the first part of
the final result (the diagonal matrix $\Sigma$); along with an
auxiliary rotation that we need to 
produce the other half of the final result (matrix $U$). But let us connect
this with previously used QR algorithm (line $2$), in order to clarify
further. \\

In the SVD literature, there is a variant called R-SVD which uses the
QR factorization as an intermediate step for SVD calculation. The
name seems to come from Golub's book \cite{golub13}, where is
introduced as a previous step to the so called R-Bidiagonalization
(the method proposed by Golub brings the original matrix $A$ to a
bidiagonal form, from where calculating SVD is easier). Putting aside
this bidiagonalization context, the main idea of using QR
factorization as an intermediate step in SVD calculation, is
summarized in equation below: \\

\begin{equation}
\label{eq:svd-qr}
A = QR = Q (\prim{U} \Sigma \trans{V}) = (Q \prim{U}) \Sigma \trans{V}
\end{equation}
\hfill

We can appreciate from equation above that the final matrix $U$ is
obtained, by composing the $\prim{U}$ matrix (from the SVD factorization
of triangular matrix $R$), with the orthogonal matrix  $Q$ (obtained from the
QR factorization of $A$). Interestingly, the matrices $\Sigma$ and $V$ from
the SVD of $R$, become the same as if one would have done SVD directly
on matrix $A$. This is essentially the idea of line $3$ from
\cref{alg:merge-svd}, which produces the diagonal matrix $\Sigma$ that
we need as final result; but it also produces a couple of additional matrices:

\begin{itemize}
  \item The orthogonal matrix $V_R^{T}$, which is discarded (let us
    recall we just care about $U$ and $\Sigma$). 
  \item The matrix $U_R$, which like in the example with $QR$
    factorization, is just an auxiliary item for producing the final
    matrix $U$ that we need (more about this on next section).
\end{itemize}
\hfill

But the side products of the ${SVD}_k$ calculation on step $3$ is
perhaps the less problematic to understand, the real trouble may come
from the matrix we are using as input for such calculation. Let us
name such matrix on the right hand side as $X$, it can be deduced from
the following requirement that we impose on the final matrix $U
= \begin{bmatrix}U_1 \mid \prim{U}\end{bmatrix}$ \footnote{The
  equality claimed on this 
  equation is not totally clear, as after the $SVD_k$ calculation of
  $X$ we drop its $V$ matrix; and the left side does not involve any
  matrix $V$. We contacted a couple of times the author (Radim
  \Rehurek) for \href{http://math.stackexchange.com/questions/1375029/merging-two-svd-factorizations-but-using-only-u-1s-1-and-u-2s-2-question-ab}{kindly
  asking for a clarification about a related equation in his thesis},
  but unfortunately we did not got a final answer.}: \\

\[
\begin{bmatrix}U_1\Sigma_1 \mid U_2\Sigma_2\end{bmatrix} = 
\begin{bmatrix}U_1 \mid \prim{U}\end{bmatrix} X
\]
\hfill

If we clear the matrix variable $X$ by multiplying each side (on the
left) by \trans{\begin{bmatrix}U_1 \mid \prim{U}\end{bmatrix}}, we get
the following (please note that we are using the matrix block
operations, which nicely behave like scalars):

\begin{equation}
\label{eq:svd-merge-x1a}
X = 
\trans{\begin{bmatrix}U_1 \mid \prim{U}\end{bmatrix}} \begin{bmatrix}U_1\Sigma_1 \mid U_2\Sigma_2\end{bmatrix} =
\begin{bmatrix}
\trans{U_1}U_1\Sigma_1 & \trans{U_1}U_2\Sigma_2 \\
\trans{\prim{U}}U_1\Sigma_1 & \trans{\prim{U}}U_2\Sigma_2
\end{bmatrix}
\end{equation}
\hfill

We need now a few additional equalities that can be inferred from the
\cref{alg:merge-svd}: \\

\begin{enumerate}
\item $U_1$ is orthogonal $\implies$ $\trans{U_1}U_1 = I$ \\
\item By construction, the set of columns from where $\prim{U}$ is
  calculated (that is, $U_2 - U_1Z$), is orthogonal to $U_1$ $\implies$
  the subspace spanned by such set is also orthogonal to $U_1$. In
  particular, any basis of that subspace is also orthogonal to
  $U_1$. Therefore $\prim{U}$ is orthogonal to $U_1$, that is,
  $\trans{\prim{U}} U_1 = 0$. \\ 
\item Using the above, and the QR calculation from line 2 of
  \cref{alg:merge-svd}, \Rehurek claims that $R =
  \trans{\prim{U}} U_2$. Such equality is not totally clear, as it seems
  as if we would be isolating $R$ from that step; however, such step
  represents an assignment, not an equation. The claim may be due may
  a property of QR calculation itself (seen as a function of matrices, rather
  than a procedure). We take it for granted \footnote{If this work is
    used for a thesis, we will seek to clarify this part though.}.
\end{enumerate}
\hfill

Using the three equalities just mentioned, the matrix $X$ from
\cref{eq:svd-merge-x1a} can be further simplified as follows: \\

\begin{equation}
\label{eq:svd-merge-x1b}
X = 
\begin{bmatrix}
\Sigma_1 & \trans{U_1}U_2\Sigma_2 \\
0        & \trans{\prim{U}}U_2 \Sigma_2
\end{bmatrix} =
\begin{bmatrix}
\Sigma_1 & Z\Sigma_2 \\
0        & R\Sigma_2
\end{bmatrix} 
\end{equation}
\hfill

It is equation \cref{eq:svd-merge-x1b} that justifies the right hand
side of step 3 in \cref{alg:merge-svd}. \\

A final note about this step, is that the ${SVD}$ routine being called
is not the same as $\func{Basecase-SVD}$ from \cref{alg:svd-dist};
while the former is a full SVD for shorter ``dense'' matrices, the
second is a truncated SVD calculation for large sparse 
ones. The dense SVD calculation is done with the standard algorithm
called Golub-Kahan-Reinsch (\cite{golub75}, \cite{golub70}), available
as a LAPACK routine \cite{lapack}); while the
truncated sparse SVD is done with the also fameous LASVD routine (later
incarnated as SVDPACKC LAS2), that Berry did from the Lanzos version
of Parlett and Simon (\cite{parlett79},\cite{simon84}). 

\subsection{Calculating the final matrix $U$}

All these auxiliary results may take us apart from our final goal, so
let us remember what it is: to produce a couple of matrices, $U$ and
$\Sigma$, which represent the merged eigen decomposition of the two
pair of matrices we received as input ($U_1$,$\Sigma_1$ and
$U_2$,$\Sigma_2$). So far, we have calculated already the diagonal
$\Sigma$; hence the remaining task is to calculate $U$. We have all
the auxiliary devices at our disposal, from previous steps of the
algorithm. \\

We began by picking orthonormal basis $U_1$, and extending it with
$\prim{U}$ in order to get a new orthonormal basis (in matrix form)
$\begin{bmatrix}U_1 \mid \prim{U}\end{bmatrix}$; such basis covers the
spanning subspaces of both $U_1$ and $U_2$. We may be tempted to
think that such matrix is the desired $U$, but the problem is that we
took the diagonal $\Sigma$ from an $SVD$ calculation; that means we
got already one orthogonal matrix for the term-space $U_R$. We need to
compose such $U_R$ with our orthonormal basis 
$\begin{bmatrix}U_1 \mid  \prim{U}\end{bmatrix}$, in order to get the final basis $U$
(due same reasons exposed in \cref{eq:svd-qr}): \\

\[
U = \begin{bmatrix}U_1 \mid  \prim{U}\end{bmatrix} U_R
\]
\hfill

But now we exploit the shape of matrix $U_R$; if we were doing full SVD
calculation in the line $3$ of \cref{alg:merge-svd}, we would have a
matrix $U_R$ of dimensions $(k_1+k_2) \times (k_1+k_2)$. But since we
are calculating the truncated SVD instead, it gets dimensions
$(k_1+k_2) \times k$. Furthermore, it can be split in two blocks $R_1$
and $R_2$ as follows: 

\[
U_R = 
\begin{bmatrix}
R_1^{k_1 \times k} \\[0.4em]
R_2^{k_2 \times k}
\end{bmatrix}
\]
\hfill

Using block multiplication in the submatrices, we can get the final
assignment from line $5$ of \cref{alg:merge-svd}: \\

\[
U \leftarrow 
\begin{bmatrix}U_1 \mid  \prim{U}\end{bmatrix} U_R =
\begin{bmatrix}U_1 \mid  \prim{U}\end{bmatrix} 
\begin{bmatrix}
R_1^{k_1 \times k} \\[0.4em]
R_2^{k_2 \times k}
\end{bmatrix} =
U_1R_1 + \prim{U}R_2
\]
\hfill

\section{Complexity and performance}

\subsection{Time complexity of the Merge-SVD algorithm}
The overall complexity of the \cref{alg:svd-dist} can be expressed in
terms of functions $\func{Basecase-SVD}$ and $\func{Merge-SVD}$; but
given the former is seen as a black box over which we have little
control, and that the main contribution of \Rehurek (from SVD 
perspective), is the \cref{alg:merge-svd}, we focus on the complexity
of that $\func{Merge-SVD}$ alone. \\

Let us review the cost of its main steps of
\cref{alg:merge-svd} individually, as a way to arriving to the the overall
complexity (we will not consider the possible parallelization or
vectorization of the basic kernel operations \footnote{A ``kernel'' in
the context of Numerical Linear Algebra, is a basic routine which is
heavily used by higher level algorithms; hence, its performance is
crucial and they are heavily optimized.}, which is usually
achieved by using standard libraries like BLAS or LAPACK): 

\begin{itemize}
  \item The matrix multiplication that produces $Z$ in line $1$, is
    done against matrices $\trans{U_1}$ (of dimensions $k_1 \times m$)
    and matrix $U_2$ (of dimensions $m \times k_2$); hence it has a
      complexity \bigO{m k_1 k_2}. \\ 

  \item The second step is dominated by the QR calculation; according
    to Golub \cite{golub13}, the complexity of a QR factorization
    based on the Gram-Schmidt process for a matrix $A^{m \times n}$,
    is \bigO{m n^2}. Applying that result to the particular case of
    line $2$, give us a complexity of \bigO{m k_2^2}. \\

   \item It seems hard to
     find reported complexities for the SVD algorithms, in the
     available literature; \Rehurek mentions that the complexity of
     the full SVD calculation from line $3$ is
     \bigO{(k_1+k_2)^3}\footnote{We could find at least one reference
       that also mentions this complexity, see
       \cite{plassman05}.}. Hence, given that the truncation factors  
     $k_1$ and $k_2$ are usually a few hundreds in the context of LSI;
     the cost of this step can be neglected. \\

   \item Finally, the complexity of the matrix operations in the last step 
     (focusing on the products only), is \bigO{mkk_1 + mkk_2}. 
\end{itemize}
\hfill

In practice the truncation factors do not vary much in LSI
applications, thus, we can simplify further. Let us assume that $k
\approx k_1 \approx k_2$, then the reported complexities in the list
above become: \bigO{m k^2}, \bigO{m k^2}, \bigO{k^3}, \bigO{m
  k^2}. Given that the number of terms $m$ will be much bigger than
the truncation factor $k$ (hundred of thousands, vs a few hundreds);
we can conclude that the overall time complexity is \bigO{m k^2}. \\

Due time constraints we did not enter into detailed memory complexity
analysis of the algorithm, but is part of our todo list (for the case
that this project evolves into a full thesis). 

\subsection{Performance with a large scale corpus}

\Rehurek used 3 different corpus to test his distributed SVD
algorithm, in the context of LSI. We focused only on the large corpus,
which was the English Wikipedia. By that time, it contained $3.2$
million documents; where 100,000 terms were chosen after removing the
stop words. That resulted in an sparse matrix of dimensions 
$100,000 \times 3,199,665$, with $0.5$Gb of non zero entries. Such
matrix can fit in memory of a modern personal computer, but as
explained earlier, the main objective of using the distributed
algorithm is to scale in time. The truncation factor $k$ was set to
$400$ during this experiment. \\

On his Phd thesis, \Rehurek reports the following wall times of the
distributed \cref{alg:svd-dist}, running on a single computer and on a
cluster: \\

\begin{itemize}
\item $8.5$ hours on a dual-core 2.53GHz MacBook Pro with 4GB RAM and
vecLib, a fast BLAS (\cite{blas})/ LAPACK (\cite{lapack}) library
provided by the vendor. \\ 

\item $2$ hours $23$ minutes on a cluster of four dual-core 2GHz Intel
  Xeons, each with 4GB of RAM, which share the same Ethernet segment
  and communicate via TCP/IP. These machines did not have any
  optimized BLAS library installed. 
\end{itemize}
\hfill

The above numbers suggest what we expected: given that the
parallelization achieved by the distributed algorithm is almost
perfect (only communication needed is on the final merge), the scaling
in time is basically linear with respect to the number of computing
nodes. \\

The \href{https://radimrehurek.com/gensim/dist_lsi.html}{gensim page}
(see \cite{gensim}) has a more up to date
experiment, which reports $5$ hours 25 minutes for a single machine;
and $1$ hour with 41 minutes for a cluster with 4 nodes (this time,
the cluster nodes got ATLAS installed, an open source BLAS/LAPACK
implementation (see \cite{atlas}). \\

\Rehurek does an additional comparison for the execution on a single
machine, by contrasting with a custom implementation of the SVD
algorithm published in \cite{zha98}, named as ZMS in his Phd
thesis. The ZMS algorithm took 109 hours, which brutally contrast with
the 2 hours 23 minutes mentioned above for the \cref{alg:svd-dist}. A
probably more fair comparison, would be against the SVD algorithm
implemented by SLEPc (\cite{hernandez07}); thought this opensource
implementation does not target specifically the LSI problem, it claims
to be distributed and highly scalable. Other comparisons with more
opensource implementations are possible: together, along with
reproducing the results published by \Rehurek with more nodes in the
cluster, are planned to be performed in a further stage of this
project. 

\section{Accuracy of the merge algorithm}

The overall numerical accuracy of any algorithm, is of crucial
relevance in the area of Numerical Analysis; in particular in the
subarea we care about in this project (Numerical Linear
Algebra). \Rehurek offers detailed and promising accuracy comparisons
between his proposal and several other available implementations,
though he does that mainly for the serial executions (single node) of
small/medium corpus sizes (not the Wikipedia experiment described in
previous section). Another pending task to verify the accuracy against
a golden standard, could be to perform experiments on a supercomputer
with enough RAM to hold the large corpus matrix; using an standard
SVD software like \cite{svdlibc}. The authors of these lines have added
such pending task, to the TODO list for further stages of
this project. \\

Despite of the above, an interesting analysis about the effect of
nested truncation that \cref{alg:svd-dist} introduces, is exposed in
\Rehurek Phd thesis \cite{rehurek11a}. Citing the work of Zha et al
(\cite{zha00}), he remarks that his distributed SVD algorithm meets the conditions to
be an stable algorithm (on the numerical sense), though no longer
exact. This should not surprise us, as the almost perfect parallelism
achieved can not come without a price: every merge of two SVD
factorizations, as produced by \cref{alg:merge-svd}, introduces some
error, in the sense that the following equality does not hold: \\

\begin{equation}
\label{eq:merge-svd-eq}
\func{SVD_k}(\begin{bmatrix}A_1 \mid A_2\end{bmatrix}) =
\func{SVD_k}(\begin{bmatrix}\func{SVD_k}(A_1) \mid \func{SVD_k}(A_2)\end{bmatrix})
\end{equation}
\hfill

In other words, calculating truncated SVD against the original input
matrices $A_1$ and $A_2$ (concatenated by columns), is not the same as
calculating the same over their truncated $SVD_k$ approximations. Let
us recall that the matrix produced by $SVD_k(A)$ is just an approximation
of original matrix $A$. \\

The precision lost by accepting as inputs rank-$k$ approximations,
instead of the original matrices, is not that bad though; it is shown in
\cite{zha00} and reused by Rehurek in \cite{rehurek11a}, that the
typical matrix $A$ that emerges from Natural Language 
Applications like LSI, ``do indeed possess the necessary structure and
that in this case, a rank-$k$ approximation of A can be expressed as a
combination of rank-k approximations of its submatrices without a
serious loss of precision''. \\

The above quote means, that the equality \cref{eq:merge-svd-eq} can
be considered to hold in practice. In strict theory we shall replace the
equality sign by an approximation sign though, as the equality sign
can be stated only on the idealistic case of exact arithmetic. Hence,
we can claim that: \\

\begin{equation}
\label{eq:merge-svd-app}
\func{SVD_k}(\begin{bmatrix}A_1 \mid A_2\end{bmatrix}) \approx
\func{SVD_k}(\begin{bmatrix}\func{SVD_k}(A_1) \mid \func{SVD_k}(A_2)\end{bmatrix})
\end{equation}
\hfill

This is in part, because the $SVD_k$ factorization of a matrix $A$ is
not actually just an approximation (as we claimed paragraphs above),
it is  ``the best'' approximation by a matrix of rank $k$, per the
Eckart-Young Theorem \cite{eckart36}. The \cref{eq:merge-svd-app},
derived from the work of \cite{zha00}, can be considered the angular
stone for the divide-and-conquer strategy of
the distributed \cref{alg:svd-dist}. Without it, we would not know if
it is valid to use the $\func{SVD}_k$ calculation itself, as a way of
combining two already calculated $\func{SVD}_k$ factorizations. A quite
interesting research path for this project, could be to seek for other
alternatives for doing the merge; or, to confirm that the scheme
proposed by \Rehurek is the optimal way of merging two truncated SVD
factorizations. 





