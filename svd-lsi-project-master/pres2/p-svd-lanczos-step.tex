\begin{frame}[plain]
	\begin{block}{}
\begin{algorithm}[H]
%
  \caption{Lanczos Tridiagonalization Step (sparse,2)}
  \setstretch{1.5}
  \SetKwInOut{Input}{Input}
  \SetKwInOut{Output}{Output}
  \DontPrintSemicolon
%
    \Input{A unit vector $\vec{q_1} \in \R{n}$ and a symmetric matrix $A^{n
        \times n}$}
%
    \Output{The sequences $\{\alpha_i\}$, $\{\beta_i\}$ and matrix $Q
      = [ \vec{q_1} | \vec{q_2} | \cdots ]$ }
%
    $k \gets 0, \beta_0 \gets 1, \vec{q_0} \gets 0, r_0 \gets \vec{q_1}$ \;
    \While {$k = 0 \lor \beta_k \ne 0$}
    {
      $\vec{q_{k+1}} \gets \dfrac{\vec{r_k}}{\beta_k}$ \;
      $k \gets k + 1$ \;
      $\alpha_k \gets \trans{\vec{q_k}}A\vec{q_k}$ \;
      $\vec{r_k} \gets A\vec{q_k} - \alpha_kq_k - \beta_{k-1}\vec{q_{k-1}}$ \;
      $\beta_k \gets \norm{\vec{r_k}}_2$ \;
    }
%
    return $(\{\alpha_i\}, \{\beta_i\}, 
             Q = [ \vec{q_1} | \vec{q_2} | \cdots ])$ \;
\end{algorithm}
\hfill
	\end{block} 
\end{frame}
