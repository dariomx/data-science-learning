\chapter{Lanczos SVD Algorithm}

Although quite useful for understanding the SVD factorization, the
implicit algorithms mentioned in the \cref{cha:svd-theory} can not be
used directly to calculate SVD in practice. This is in part due the
intrinsic errors associated with using a finite-precision device like
a computer (while the theorems we proved the existence of SVD, assumed
infinite precision). There is a whole area of Mathematics, called
Numerical Analysis, which targets the proper translation of
theoretical algorithms into numerical ones; which can produce accurate
results on a computer. A quite notable subarea, Numerical Linear
Algebra, has received special attention over decades of research. The
previous chapter mentions some of the main aspects to consider, with
this regard. \\

Therefore, in order to calculate SVD in practice, more specialized
theorems that consider the limitations of computers are required; such
theorems allow one to create particular algorithms, which possess the
desired qualities: not only accuracy but high performance as well.
This is specially true in today's world, where the scale of the
matrices to analyze by far exceeds those used in the past. The
particular case of Latent Semantic Indexing (LSI), the particular
application we chose to restrict our study of SVD, illustrates very
well this trend in the change of scale: original papers about LSI used
a few thousands of documents, while today applications can easily
reach millions, hundreds of millions or even more (like the Oracle
product mentioned on the introductory chapter, which considers as
documents ``posts'' in social networks). \\

In this chapter we will describe the serial algorithm that is most
commonly used, for solving the SVD factorization that comes from the
LSI problem. A particular characteristic of this algorithm, at least
in the form presented here, is that is implicitly assumed that the
matrix and auxiliary data fit in the computer memory; therefore, it
will have certain limitations in terms the size of the matrix to
factorize (at least in commodity hardware, which is our focus).  The
\cref{cha:svd-dist} describes another distributed algorithm, which does allow
to spread the matrix in a cluster of commodity computers. \\

Although the algorithm presented in this chapter is essentially
serial, some of the routines it uses can accept parallelism and take
advantage of either multi-core computers with shared memory; or
vectorial processors. We mention those details in the last section as
well. 

\section{SVD as an eigen problem}
\label{sec:svd-lanczos-eigen}

Aiming to calculate numerically the SVD factorizations, made
researchers reformulate that problem as the quite related
eigen decomposition (or eigenproblem). Such problem consists in
finding, for an square matrix $A$, the eigenvalues and
eigenvectors. If we arrange the eigenvectors in an orthogonal matrix
$Q$ and the eigenvalues in a diagonal matrix $\Lambda$, the eigen
problem can be restated as the following factorization: \\

\[
A = Q \Sigma \trans{Q}
\]
\hfill

In order to see the connection between the SVD and the eigenproblem,
we need to recall the gramian matrix $\trans{A}A$ from the theory
chapter. It was the gramian, which provided the matrix $V$ on the
first place; because the vectors 
$\vec{v}$ were its eigenvectors (see \cref{cha:svd-theory}. Finding
the matrix $V$ then, can be thought as the eigenproblem for matrix
$\trans{A}A$; which can be stated as finding its diagonal
factorization $\suchthat$: 

\begin{equation}
\label{eq:eigenprob-doc}
\trans{A}A = V\Sigma^2\trans{V}
\end{equation}
\hfill

But the same is true for matrix $U$, if we now consider the matrix
$A\trans{A}$, which can be diagonalized if one finds its eigenvectors
and place them into the matrix $U$ (the eigenvalues are the same as
the gramian): 

\begin{equation}
\label{eq:eigenprob-term}
A\trans{A} = U\Sigma^2\trans{U}
\end{equation}
\hfill

It is the second eigenvalue problem equivalence, that is used for this
distributed algorithm of chapter \cref{cha:svd-dist}. Per the SVD
factorization $A = U\Sigma\trans{V}$, if we have the original matrix
$A$, plus the diagonal $\Sigma$ and the matrix $U$;  we can
reconstruct the matrix $V$ (if required): 

\[
\trans{V} = \inv{S} \trans{U} A = P A
\]
\hfill

The matrix $P = \inv{S} \trans{U}$ is called the projection matrix, and
is used in LSI for ``folding-in'' new document vectors \vec{x}, by
calculating $P\vec{x}$; that is, the matrix $P$ is used as a
predictive (rather than descriptive) model, to predict where the
position of document \vec{x} will be in the latent space. \\

For this chapter though, we could use either eigenproblem from
\cref{eq:eigenprob-doc} or \cref{eq:eigenprob-term}. Actually, the literature
originally reported the former, perhaps due the early shape of the
matrices used for LSI (more terms than documents). Today's LSI
applications have much more documents than terms, but still these
early algorithms are useful, as we will see in \cref{cha:svd-dist}
(where the original matrix is split into several submatrices, which do
have the shape expected by Lanczos algorithm that we document here).


\section{Derivation of the serial algorithm}

The algorithm to numerically solve the SVD problem, that we chose for
this report, is essentially the one published by Berry on his PhD
thesis (\cite{berry91}). Despite of being more than 20 years old, we can
tell that it is still widely used, in particular by LSI software. A
big part of the opensource LSI implementations that we found (see
introductory chapter), refer to either the Fortran77 SVDPACK
(\cite{svdpack}), to its C incarnation SVDPACKC (\cite{svdpackc}), or
to its even newer skin SVDLIBC (\cite{svdlibc}). All of them are
essentially the same algorithm that Berry published in his PhD
thesis. \\

We will proceed to derive the algorith in the next sub-sections. A
cautionary warning about the level of detail presented is appropriate:
although we would like to offer the same level of detail and
technicality than the theory \cref{cha:svd-theory}, time constraints
for the delivery of this report forced us just to omit the theorems
and its proofs. A pending task, for this project to evolve into a full
Msc. thesis, would be to achieve the same level of formality than
\cref{cha:svd-theory}, indeed. Such exercise is actually required, if
one pretends to offer an innovation to the problem of efficiently
compute the SVD for LSI problem. Let us consider it a pending task
then, for the time being. \\

Since we established the equivalence of the SVD problem, to that of the
eigenproblem for gramian matrix $\trans{A}A$, is important that we
keep in mind such alternate formulation the next sections to come. From now
on, to the end of this section, our goal will be to find the
eigenvalues and eigenvectors of a symmetric matrix; assuming that the
calculations are to be performed on a computer with finite
precision. We will find with simple methods, and evolve them until we
reach the level of sophistication that we require for a practical SVD
algorithm. \\

\subsection{The Power Method}

According to Golub \cite{golub00}, quoting Householder, the power
method has its origin at the work of Müntz in 1913
\cite{muntz1913}. The method is the simplest algorithm for solving the
eigenproblem; it basically consists in picking carefully a vector,
and then apply the matrix $A$ iteratively until it converges to an
eigenvector. And not to any eigenvector, but precisely to the one with
largest eigenvalue (in absolute value). The following pseudocode is
taken from Golub \cite{golub13}: 

\begin{algorithm}
  \label{alg:power-method}
  \caption{The Power Method}
%
  \setstretch{1.5}
  \SetKwInOut{Input}{Input}
  \SetKwInOut{Output}{Output}
  \DontPrintSemicolon
%
    \Input{A unit vector $\vec{q_0} \in \R{n}$ and a symmetric matrix $A^{n
        \times n}$}
%
    \Output{The tuple $(\lambda_k,\vec{q_k})$ which is expected to
      approximate an eigenpair $(\lambda,\vec{q})$ of $A$}
%
    \For {$i = 1,2,\dots,k$}
    {
      $\vec{z_k} \gets A \vec{q_{k-1}}$ \;
      $\vec{q_k} \gets \dfrac{\vec{z_k}}{\norm{\vec{z_k}}_2}$ \;
      $\lambda_k \gets \trans{\vec{q_k}} A \vec{q_k}$ \;
    }
%
    return $(\lambda_k, \vec{q_k})$ \;
\end{algorithm}
\hfill

One immediate trick that is detected, is that we are not giving a
precise value for $k$, the number of iterations; this is because we do
not really know how many in advance, though we know how ``fast'' we
can reach convergence, more about this in a minute. Though the
algorithm looks trivial, a powerful theorem justifies why it
works. Golub mentions the  conditions which are required for its
convergence: the 
maximum eigenvalue of $A$ must be unique (no repetition), and the 
initial vector $q_0$ is not ``deficient''  (its component on the
direction of the eigenvector with maximum value must not be zero). A
proof of convergence/correctness can be consulted in \cite{golub13}. \\\

Golub also mentions the computable error bounds of this method.
The real eigenvalue and eigenvectors of $A$ will satisfy the equation
below: 

\[
A \vec{q} = \lambda \vec{q}
\]
\hfill

But accepting the fact that a computer will not product exactly the
eigenvalue nor the eigenvector, we can at least see how close we are
in meeting above condition. That is, we can calculate the error $\delta$: 

\[
\norm{A \vec{q_k} - \lambda_k \vec{q_k}}_2 = \delta
\]
\hfill

Golub shows that there is an eigenvalue $\lambda$ that satisfies
$\abs{\lambda_k - \lambda} \le \sqrt{2}\delta$; which is a way to tell
that we can really approximate an actual eigenvalue, as long as we are
capable of reproducing its defining property with good accuracy (which
in turn, will depend on how many iterations we make). \\

Alright, so we know how to calculate one eigenpair; why not
calculating them all? We may be tempted now to recall the geometric
proof of SVD (see \cref{cha:svd-theory}), and consider the following
procedure for finding all the eigenpairs (assuming preconditions met):

\begin{enumerate}
\item Pick carefully initial vector. 
\item Apply \cref{alg:power-method} to find the first
  eigenpair. 
\item Obtain the hyperplane that is orthogonal to the first
  eigenvector found, and repeat recursively the procedure until
  we have all the desired eigenpairs\footnote{The third step is
    usually called ``deflation'' (see \cite{golub13}), when mentioned
    in the context of the matrix, as it is reduced to dimensions $(n-1)
    \times (n-1)$.}.
\end{enumerate}
\hfill

The problem with this procedure, also exposed by Golub in his proof of
correctness, is that the rate of convergence depends on
$\abs{\frac{\lambda_2}{\lambda_1}}^k$; where $\lambda_2$ is the second
largest eigenvalue in absolute value. Thus, unless there is a
considerable gap between first and second largest eigenvalues of $A$,
the Power Method will converge quite slowly. That makes it unsuitable
for practical purposes, at least in the standalone version we just
presented. Further sections will show how it can evolve to overcome
this limitation. 

\subsection{The Rayleigh-Ritz Method}

The method presented in this subsection is not really an
step forward from the Power Method, but rather a parallel development
(both will be merged in the Lanczos Algorithm of further
subsections). It is actually an auxiliary tool that many eigenproblem
solvers need; not necessarily for symmetric matrices (although we
still assume that, it order to maintain our desired scope). \\ 

Suppose that, in order to find the eigenpairs \footnote {An eigenpair
  is the tuple $(\lambda_i,\vec{v})$ of an eigenvalue and its
  corresponding eigenvector} of a given matrix $A$,
we generate a sequence of matrices ${ W_k }$ which contain
progressively better approximations of such eigenpairs. A common
problem for any procedure that goes that way, is how to ``extract''
the actual eigenvectors from such subspace (the eigenvalues are the
same, so those do not require further calculations). The
Rayleigh-Ritz \footnote{Leissa argues that the method should not
  really be attributed to Rayleigh but only to Ritz, (see
  \cite{leissa05}).} method addresses precisely this common need. \\

Before providing the pseudocode, let us explain a bit better what we
mean by having ``calculated subspaces'' $W_k$; as that is a rather
vague expression (though is quite common in the literature). What we
really mean, is that we have a \emph{characterization} of the
subspace; which is nothing more than a basis for it. The vectors of
such basis are arranged as columns of the matrix $W_k$, and then, we
are basically asking for the eigen decomposition of that matrix. \\

Does not the above sound a bit circular? We start with the generic
problem of finding eigenvalues and eigenvectors of symmetric matrix
$A$; then we calculate through an iterative process another matrix
$W_k$, which contains the basis of a subspace that we know has good
approximations to the eigenpairs of our original matrix $A$. Then, we
proceed to solve the eigenproblem for that new matrix $W_k$ ... looks
like we finish right where we began! Of course that, though not
mentioned always in literature, the intuitive idea is that the new
matrix $W_k$ is a less generic than $A$. It is expected to have certain
qualities that make the solution of its eigenproblem an easier task
(compared to solving that for original matrix $A$).  \\

Having clarified a bit the main idea of subspace eigenproblem solvers,
let us continue to list the pseudocode for the Rayleigh-Ritz Method; which
offers a way to ``extract'' the eigenvectors of original matrix $A$,
out of the approximation matrix $W_k$. We based our procedure in
\cite{jia01}: 

\begin{algorithm}
  \label{alg:ritz}
  \caption{The Rayleigh-Ritz Method}
%
  \setstretch{1.5}
  \SetKwInOut{Input}{Input}
  \SetKwInOut{Output}{Output}
  \DontPrintSemicolon
%
    \Input{Approximation subspace matrix $W_k$, symmetric matrix $A$}
%
    \Output{Set of desired (approximated) eigenpairs}
%
    $B \gets \trans{W} A W $ \;
%
    \For {each desired eigenpair $(\lambda_i,\vec{v_i})$ of $A$}
    {
%
      Solve eigen equation $B\vec{x_i} = \tilde{\lambda_i} \vec{x_i}$ (where
      $\tilde{\lambda_i} \simeq \lambda_i$) \;
%
      $(\tilde{\lambda_i}, \avec{v_i}) \gets (\lambda_i, W\vec{x_i})$,
      where $\avec{v_i} \approx \vec{v_i}$ \;
    }
%
    return $\{(\tilde{\lambda_1},\avec{v_1}),(\tilde{\lambda_2},\avec{v_2}),\cdots,\}$ \;
\end{algorithm}

If $W_k$ was the orthogonal matrix with the eigenvectors of $A$, then
matrix $B$ would be diagonal (containing the eigenvalues). As $W_k$ is
rather an approximation to such matrix, is usually the case that is
something close to a diagonal (like a tridiagonal or bidiagonal); from
there comes the fact that calculating its eigenpairs, is much easier
than for original matrix $A$. \\

Probably the less intuitive step from the algorithm is the assignment
$\avec{v_i} = W\vec{v_i}$; but is not hard to prove its validity: 

\begin{align*}
\setstretch{10}
& &B\vec{x_i} = \tilde{\lambda_i} \vec{x_i} \\\\
& \iff &(\trans{W} A W)\vec{x_i} = \tilde{\lambda_i} \vec{x_i} \\\\
& \iff &\trans{W} A (W\vec{x_i}) = \tilde{\lambda_i} \vec{x_i} \\\\
& \iff &A (W\vec{x_i}) = \tilde{\lambda_i} (W \vec{x_i}) \\\\
& \therefore & W\vec{x_i} \text{ is an (approx.) eigenvector of $A$} \xqed
\end{align*}
\hfill

The vector \vec{v_i} is called a Ritz vector, and we will refer to it
in such a way when we review the complete Lanczos algorithm.\\

For further details of convergence or error analysis, please refer to
Jian \cite{jia01}. 


\subsection{The Lanczos Tridiagonalization Step}

Golub explains in \cite{golub13} that one of the problems with the
Power Method, is that it does not take advantage of the previously
calculated information. During the iterations of the Power Method, say
until step $k$, we have calculated already the set of vectors
$K(A,q_0,k) = \{A\vec{q_0},A\vec{q_1},\dots,A\vec{q_k}\}$; still, they
are not used 
at all when looking for an estimate of the eigenvector. Such
limitation is addressed by the Lanczos Process, named after its
creator in 1950 (\cite{lanczos1950}). The subspace spanned by the
$K(A,q_0,k)$ is called Krylov Subspace of order $k$ \footnote{The
  concept itself of Krylov Spaces is thanks for 
  Krylov and dates back to 1931 (see \cite{krylov1931})}, which is why
the Lanczos Process is usually cataloged as a Krylov Subspace
method.  \\

Going back to Lanczos, this subsection will only explain the iterative
step (called Lanczos Tridiagonalization Step). It works as follows; let us
suppose that we have an square symmetric matrix $A^{n \times n}$, and
that we want a few of its biggest eigenvalues (as it is the case in LSI
applications) \footnote{The Lanczos Process can also calculate a few
  of the smallest eigenvalues, but we are not interested in such case
  for LSI applications.}. Each iteration $k$ of the algorithm generates a
tridiagonal matrix $T_k \in \R{k \times
  k}$ \footnote{Matrices with the middle, upper and lower
  diagonals.}, and the whole sequence ${T_k}$ is  progressively 
approximating the biggest eigenvalues of the original matrix
$A$. \\

There are several ways of stating the algorithm for the Lanczos
Tridiagonalization Step, the following is taken from Golub \cite{golub13};
though it is not the the most numerically stable. That honor
corresponds to the ones created by Paige
(\cite{paige71},\cite{paige76}); we preferred Golub's one for our
exposition, aiming to have an easier introduction to the procedure: 

\begin{algorithm}
  \label{alg:lanczos-step}
  \caption{The Lanczos Tridiagonalization Step}
%
  \setstretch{1.5}
  \SetKwInOut{Input}{Input}
  \SetKwInOut{Output}{Output}
  \DontPrintSemicolon
%
    \Input{A unit vector $\vec{q_1} \in \R{n}$ and a symmetric matrix $A^{n
        \times n}$}
%
    \Output{The sequences $\{\alpha_i\}$, $\{\beta_i\}$ and matrix $Q
      = [ \vec{q_1} | \vec{q_2} | \cdots ]$ }
%
    $k \gets 0, \beta_0 \gets 1, \vec{q_0} \gets 0, r_0 \gets \vec{q_1}$ \;
    \While {$k = 0 \lor \beta_k \ne 0$}
    {
      $\vec{q_{k+1}} \gets \dfrac{\vec{r_k}}{B_k}$ \;
      $k \gets k + 1$ \;
      $\alpha_k \gets \trans{\vec{q_k}}A\vec{q_k}$ \;
      $\vec{r_k} \gets A\vec{q_k} - \alpha_kq_k - \beta_{k-1}\vec{q_{k-1}}$ \;
      $\beta_k \gets \norm{\vec{r_k}}_2$ \;
    }
%
    return $(\{\alpha_i\}, \{\beta_i\}, 
             Q = [ \vec{q_1} | \vec{q_2} | \cdots ])$ \;
\end{algorithm}
\hfill

The \cref{alg:lanczos-step} is essentially applying Gram-Schmidt process,
but only against the last two vectors. Golub derives the algorithm
from a relation between tridiagonalization, and the QR factorization of
the matrix formed by vectors $K(A,q_0,k)$; see \cite{golub13} for
further details. \\

Golub goes even further in the cited book, and proves the following
properties about \cref{alg:lanczos-step}. We will omit the theorem
statement, and just comment directly its results: \\

\begin{itemize}
  \item The algorithm runs until $k = m = rank(K(A,q_0,k))$. This contrasts 
    with the unknown number of steps of the Power Method
    (\cref{alg:power-method}). \\

  \item For $k = 1:m$ we have $AQ_k = Q_kT_k + \vec{r_k}\trans{e_k}$,
    where $Q = [\vec{q_1} | \cdots | \vec{q_k} ]$ has orthonormal
    columns that span the Krylov subspace $K(A,\vec{q_1},k)$, and $e_k
    = I_n(:,k)$ (the $k$ column of the identity matrix). This
    justifies the orthogonalization step of the algorithm (line 6),
    which only considers the last two vectors; whether that is enough to
    guarantee that all the $\vec{q}$\apos{s} will be orthogonal is
    certainly not evident, and gets proved on the same theorem. \\

  \item The matrix $T_k$ has tridiagonal shape, that is: \\
    \[
    \begin{bmatrix}
      \alpha_1 & \beta_1 & \cdots      & 0          \\
      \beta_1  & \ddots  & \ddots      & \vdots     \\
      \vdots   & \ddots  & \ddots      & \beta_{k-1} \\
      0        & \cdots  & \beta_{k-1} & \alpha_k
    \end{bmatrix}
    \]
    \hfill

    This shape allows us to calculate its eigenvalues with much less
    effort than for original matrix $A$ (which was the whole
    motivation on the beginning). There are several options for such
    calculation, but we will consider only the (implicit) QL Algorithm
    (see \cite{dubrulle71}), as that is the one used by Berry for his
    famous routines in the context of LSI (see further
    subsections). 
\end{itemize}
\hfill

In addition to the above properties, which kind of guarantee the
``correctness'' of the \cref{alg:lanczos-step} (to some extent); Golub
also cites in \cite{golub13} another theorem that establishes the
approximation quality of matrix $T_k$ as a function of $k$. This is
the result that justifies our original claim that the sequence of
matrices $\{T_k\}$, approximates better the eigenvalues of $A$ as $k$
increases. \\

Finally, Golub adds in \cite{golub13} that not everything is flakes
and honey with this 
algorithm; the orthogonality that we expect on vectors \vec{q}\apos{s}
is at jeopardy as $\tilde{\beta_k}$, the numerical approximation of
$\beta_k$, becomes really small; this is because that implies the
cancellation of $\vec{r_k}$). Main credit of this result goes again to
Paige (\cite{paige71},\cite{paige76}), and we will come back to 
it on next subsection, when we show the full Lanczos Algorithm.




\subsection{The Single-Vector Lanczos Algorithm}

We are armed now with all the required tools to present the main
algorithm that is used for SVD, in the context of LSI (at least in the
serial form). As mentioned on the introductory section of this
chapter, the algorithm exists thanks to Berry
(\cite{berry92},\cite{berry95}); we should probably present our
respects to Berry in this moment, as he worked for more than a decade
around the particular problem of solving efficiently the SVD/LSI
problem (and in essence, today's applications still use his
contributions). \\

The \cref{alg:lasvd} we present below is not the final one, as there are more
practical considerations to cover at the end of this subsection; but it is
easier to present this simplified version, as it has all the main
ingredients. We can see in particular, how it combines Lanczos
Tridiagonalization Step (\cref{alg:lanczos-step}) (which implicitly
uses the Power Method \cref{alg:power-method}), with the Rayleigh-Ritz
Method (\cref{alg:ritz}). \\

We also take the opportunity to come back to our original context,
where $A$ is a large and sparse matrix coming from an LSI problem. The
pseudocode is based on \cite{berry92}, filling additional details
from the C code of the implemented routine (LAS2) \footnote{Originally
the routine was coded in Fortran77, but we found more comfortable to
check the C port instead; both made by Berry, by the way.}.

\begin{algorithm}
  \label{alg:lasvd}
  \caption{The Single-Vector Lanczos Algorithm}
%
  \setstretch{1}
  \SetKwInOut{Input}{Input}
  \SetKwInOut{Output}{Output}
  \DontPrintSemicolon
%
    \Input{A matrix $A^{m \times n}$ and a truncation factor
    $k$}
    \BlankLine
%
    \Output{The $k$ singular values and its associated right singular
      vectors of $A$ \footnote{The eigenvectors of matrix $\trans{A}A$ are called
        right singular vectors of $A$, while those of $A\trans{A}$ are the
        left singular vectors.} (which are the first $k$ eigenpairs of
      symmetric matrix $\trans{A}A$). Both are numeric approximations. } 
    \BlankLine
    \BlankLine
%
    Use Lanczos Tridiagonalization step \cref{alg:lanczos-step} to
    generate a family of symmetric tridiagonal matrices, $\{ T_j \} (j
    = 1,2, \dots, c) \suchthat c > k$. Note that these matrices
    approximate the eigenvalues of symmetric matrix
    $\trans{A}A$\ (which happen to be the singular values of $A$). \;
    \BlankLine
    \BlankLine
%
    \strut Compute the eigenvalues and eigenvectors of $T_k$ using the
    (implicit) QL Method. \;
    \BlankLine
    \BlankLine
%   
    For each computed eigenvalue $\lambda_i$ of $T_k$ (hence of gramian matrix
    $\trans{A}A$), calculate the associated unit eigenvector $\vec{z_i}$
    such that $T_k\vec{z_i} = \lambda_i\vec{z_i}$. \;
    \BlankLine
    \BlankLine
% 
    For each calculated eigenvector $\vec{z_i}$ of $T_k$, compute the Ritz
    vectors $v_i = Q_c\vec{z_i}$ as an approximation to the
    $i$-th eigenvector of $\trans{A}A$ (hence, to the right singular
    vectors of $A$). Note that the matrix $Q_c$ is a side product of
    the first step. \; 
    \BlankLine
    \BlankLine
%
    return $(\{\lambda_1,\lambda_2,\cdots,\lambda_k\},
            \{\vec{v_1},\vec{v_2},\cdots,\vec{v_k}\})$
\end{algorithm}
\hfill

Although complete in appearance, \cref{alg:lasvd} still has a serious
numerical issue: the potential loss of orthogonality in the vectors of
matrix $Q_c$. To solve that problem, we could reorthogonalize all
vectors at every execution of \cref{alg:lanczos-step}; but that would
be kind of brute force, and eliminate the advantages of the whole
proposal. A clever approach, selective reorthogonalization, was
selected by Berry in order to complete his master-piece: the LASVD/LAS2
routine \footnote{Berry actually proposed four different methods of
  calculating SVD, for the LSI problem; LAS2 (descendant of LASVD)
  routine is just one of them. But it seems the fastest, and it was the
  only one ported to the new skin of Berry's SVDPACKC,
  which is SVDLIBC \cite{svdlibc}. Interestingly though, Berry
  mentions in \cite{berry91} that LASVD is suitable only for low to
  medium precision in the singular values. A pending task then, is to
  confirm of modern incarnations still have such limitation.}. \\

The selective reorthogonalization approach, as explained by Golub in
\cite{golub13}, is inspired on the error analysis made by Paige
  \cite{paige71}. Peige shows that the most recently computed vector
  \vec{\tilde{q_{k+1}}}, tends to have a non trivial and unwanted
  component in the direction of the already converged Ritz vectors
  \footnote{Recall that the Ritz vectors approximate the eigenvectors
    of the gramian matrix of $A$, hence the singular vectors of
    $A$}. Therefore, we do not need to re-orthogonalize against all
  the previously calculated vectors, rather use only the already
  converged ones. \\

Such adjustment is done during the Lanczos step
(\cref{alg:lanczos-step}), using a criteria devised by Parlett et al
\cite{parlett79}, which allows one to: know when a Ritz vector is
converged. \\

Berry does not include a final pseudocode of his LASVD
routine (inspired on the LANSOS routine from Parlett, Simon et
al). The routine eventually got renamed as LAS2 and made its way into
the famous SVDPACK (Fortran77) and SVDPACKC libraries; and more
recently in the modern version called SVDLIBC. It is the latest, which
is currently used by several LSI applications. 








\section{Profiling and Parallelization}

Berry does some interesting profiling about the \cref{alg:lasvd}, in
his PhD thesis \cite{berry91}. He was specially interested in
parallelizing such algorithm, along with other three methods he
proposed. The numbers he reported used a term-document matrix of 
$5831 \times 1033$; he tested in the medium size Alliant FX/80
computer (with 8 processors), as well as the supercomputer
Cray-2S/4-128 (with 4 processors). The Cray computer was able to
deliver, in theory, 1.9 Gigaflops; as opposed to the 200 megaflops of
the Alliant computer. The wall times he reports may no longer be
relevant for today's LSI applications, as the data and the computers
have changed much in the last 3 decades. But the profiling he did is
still relevant, and actually we could not find a more up to date
experiment (it would be an interesting exercise to do one). \\

\subsection{Linear Algebra Kernels: BLAS and LAPACK}

The \cref{alg:lasvd} was implemented in the tradition of Linear
Numerical Algebra; one never reinvents the wheel but reuses existing
standard libraries (called \emph{kernel} routines). This is specially
important to avoid introducing 
numerical errors; as it would be quite impossible that all the people
knew the specialized details which are required to produce
high-quality routines. A bonus that scientific programmers get by
using these standards, is the potentially parallel implementation of
the kernels (routines) being used. Today's standard are: 

\begin{itemize}
\item BLAS (Basic Linear Algebra Subroutines) \cite{blas}: which
  originated in the Fortran77 world, but now have bindings to many
  modern languages. They are classified in three levels: level 1 for
  vector-vector operations, level 2 for matrix-vector operations and
  level 3 for matrix-matrix operations. These routines are highly
  specialized for particular processors/architectures, taking
  advantage of the memory hierarchy, multi-cores, vectorial
  capabilities of processors, custom assembler instructions,
  etc. (\footnote{The ability to execute the 
    same basic operation against several data; known
    examples out of the super-computers world are the Intel SSE
    features (see \cite{sse}).}). \\
%
\item LAPACK (Linear Algebra Package): is the modern incarnation of
  the old libraries Linpack \cite{linpack} and Eispack \cite{eispack},
  which implemented several numerical algorithms of Linear Algebra in
  general, and in particular for solving the eigenproblem. LAPACK's original
  goal was to make efficient implementation of those libraries, by
  having specialized and highly optimized code for specific
  architectures. It is built on the lower level BLAS library, but it
  also has its own optimizations for many hardware vendors.
\end{itemize}
\hfill

By the time Berry wrote his PhD thesis, LAPACK was not yet the
standard, so he used Eispack instead; BLAS was available since
then. He used the optimized implementations of these libraries for the
two computers described above. The original implementation of Berry
used the routines mentioned in \cref{tab:lasvd-kernels} (the list is
not exhaustive, but includes the most relevant ones, performance-wise): \\

\begin{table}[!h]
\caption{Original BLAS and EISPACK routines used by \cref{alg:lasvd}}
\label{tab:lasvd-kernels}
\begin{center}
\begin{tabular}{|c|c|c|}
\hline
Routine & Library & Description \\
\hline
\hline
SPMXV & BLAS level 2 & Sparse matrix-vector multiplication \\
\hline
IMTQL2 / TRED2 & EISPACK & Implement the (implicit) QL Algorithm. \\
\hline
DAXPY & BLAS level 1 & $\vec{x} \gets \gamma \vec{x} + \vec{y}$ \\
\hline
DAXPY & BLAS level 1 & $\vec{x} \gets \vec{y}$ \\
\hline
DDOT & BLAS level 1 & $\vec{x} \cdot \vec{y}$ \\
\hline
\end{tabular}
\end{center}
\end{table}

\subsection{The two hot spots: SPMXV and IMTQL2}

The \cref{tab:lasvd-prof} shows the results obtained by Berry; he
measured 
the speedup of the subroutines when incrementing the number of
processors from 1 to 8 on the Alliant FX/80 computer (unfortunately he
did not include speedups details for the Cray-2S/4-128). In addition,
he includes the results of his profiling, by showing the percentage of
the total time that each routine consumed. \\

\begin{table}[!h]
\caption{Original profiling and speedups for \cref{alg:lasvd}}
\label{tab:lasvd-prof}
\begin{center}
\begin{tabular}{|c|c|c|c|c|}
\hline
& \multicolumn{2}{|c|}{Alliant FX/80} & \multicolumn{2}{|c|}{Cray-2S/4-128} \\
\hline
Routine & Speedup & \%CPU Time & Speedup & \%CPU Time \\
\hline
\hline
SPMXV & 3 & 27\% & - & 72\% \\
\hline
IMTQL2 & 4.3 & 14\% & - & 12\% \\
\hline
DAXPY & 5 & 17\% & - & - \\
\hline
DCOPY & 3.6 & 20\% & - & - \\
\hline
DDOT & 7.7 & 2\% & - & - \\
\hline
\hline
\end{tabular}
\end{center}
\end{table}

The above numbers quickly tell us that the routine SPMXV is the main
bottleneck, and the one which would benefit more from the
optimizations in optimized BLAS libraries. This multiplication comes
from the Lanczos Tridiagonalization Step (\cref{alg:lanczos-step}),
while calculating the product of the input matrix $\trans{A}A$ by the
vectors \vec{q_k}. The fact that such matrix is never referred in a
matrix-matrix operation, but only matrix-vector ones, is the main
reason for claiming that Berry's algorithm is suitable for sparse
matrices. Internally, the routine SPMXV may exploit the 
format of the sparse gramian matrix in order to perform wise
optimizations. \\

The BLAS level 1 routines seem to have a quite different performance
across the two tested computers, and Berry mentions in \cite{berry91},
that it was due a synchronization required on the Alliant
computer. Still, all these routines have great speedups with several
processors; which is not surprising given their SIMD \footnote{Single
  Instruction Multiple Data, a type of parallelism.} nature. Together with the
SPMXV routine, the BLAS level 1 and level 2 kernels are likely to
represent beyond 50\% of the total time. Simply installing an
optimized BLAS library should suffice to give our algorithm a good
parallel boost. \\

The second candidate for enjoying the parallelization is the higher
level routine IMTQL2, which calculates the eigenvalues and
eigenvectors of the tridiagonal matrix produced by
\cref{alg:lanczos-step}; it uses the Implicit QL Algorithm
\cite{dubrulle71} for such purpose. Berry claims in \cite{berry91} and
\cite{berry92}, that such 
routine could clearly use parallel techniques; and the speedups
reported in \cref{tab:lasvd-prof} seem to confirm such claim indeed. However,
newer tests need to be performed with new hardware, new data and new
libraries; in order to see if this routine is still worth to be
parallel. Expectation is that the matrix-vector and vector-vector
operations, still dominate whole performance of the algorithm. 

\subsection{SVDLIBC: a history of lost parallelism}

When one reads from Berry's papers about parallel SVD for
large sparse matrices, that the \cref{alg:lasvd} accepts parallelism
indeed; one takes for granted that modern incarnations inherited this
feature. We will proceed to show that such assumption is incorrect. \\

The original Fortran77 
implementation of Berry was in SVDPACK \cite{svdpack}, which used
directly BLAS routine SPMXV, as well as Eispack IMTQL2. This allowed
transparent parallelism, as long as the environment had installed the
optimized vendor libraries. \\

But, perhaps motivated by the profiling results we showed in
\cref{tab:lasvd-prof}, Berry changed the implementation in the C incarnation
SVCPACKC \cite{svdpackc}; he stopped using directly the BLAS routine
SPMXV, and instead accepted it as a user parameter (aiming to
achieve higher flexibility, we presume). The other change he did, was
to include directly a serial implementation of IMTQL2; this decision
was crucial, as it prevented his \cref{alg:lasvd} from enjoying
parallelization in step 2. \\

Old ``Fortranish'' conventions are difficult to grap by new
generations of programmers, and this motivated the rewrite of SVDPACKC
into a modern skin called SVDLIBC \cite{svdlibc}. Although is mostly a
change of style, it made another implicit serialization: the
matrix-vector operations that were previously accepted as parameters,
are now included with a serial implementation. In essence then, users
of SVDLIBC are using a serial implementation of Berry's parallel
\cref{alg:lasvd}. This seems like an unfortunate accident, as todays
computers (even personal ones), usually have multi-core and vectorial
capabilities. \\

Above finding is specially relevant for us, as the SVDLIBC
implementation plays the role of the \func{Basecase-SVD} function in
the distributed SVD \cref{alg:svd-dist} (is used inside
\func{SVD-Node} function, see also
\cref{alg:svd-dist-node}). \Rehurek, the author, does 
not offer profiling reports to see how much time is spent in
\func{Basecase-SVD} function; but
it is suspected to be a significant part. Such function can definitely
benefit from using an optimized version of SVDLIBC (which internally
takes advantage of BLAS/LAPACK installations). Currently, \Rehurek
reports that only \cref{alg:merge-svd} takes advantage of vendor
BLAS/LAPACK implementations \footnote{\Rehurek codes the
  \cref{alg:merge-svd} himself, using NumPy routines \cite{numpy}; which
  definitely takes advantage of the optimized BLAS/LAPACK
  kernels.}. We added this task to our TODO list, in case this project
evolves into our Msc. thesis.

