\begin{frame}[plain]
	\frametitle{[Curran02] Thesaurus construction overview}
	\begin{block}{}
    Vector-space thesaurus extraction can be separated
    into two independent processes. The first step extracts 
    the contexts from raw text and compiles them into a 
    vector-space statistical description of the contexts 
    each potential thesaurus term appears in.
	\end{block} 
	\begin{block}{}
    The second step in thesaurus extraction performs
    clustering or nearest-neighbour analysis to determine which terms
    are similar based on their context 
    vectors. For nearest-neighbour measurements
    we must define a function to judge the similarity between two
    context vectors (e.g. the cosine measure) 
    and a function to combine the raw instance frequencies for each
    context relation into weighted vector 
    components.
	\end{block} 
\end{frame}
