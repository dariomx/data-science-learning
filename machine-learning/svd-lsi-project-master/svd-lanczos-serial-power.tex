\subsection{The Power Method}

According to Golub \cite{golub00}, quoting Householder, the power
method has its origin at the work of Müntz in 1913
\cite{muntz1913}. The method is the simplest algorithm for solving the
eigenproblem; it basically consists in picking carefully a vector,
and then apply the matrix $A$ iteratively until it converges to an
eigenvector. And not to any eigenvector, but precisely to the one with
largest eigenvalue (in absolute value). The following pseudocode is
taken from Golub \cite{golub13}: 

\begin{algorithm}
  \label{alg:power-method}
  \caption{The Power Method}
%
  \setstretch{1.5}
  \SetKwInOut{Input}{Input}
  \SetKwInOut{Output}{Output}
  \DontPrintSemicolon
%
    \Input{A unit vector $\vec{q_0} \in \R{n}$ and a symmetric matrix $A^{n
        \times n}$}
%
    \Output{The tuple $(\lambda_k,\vec{q_k})$ which is expected to
      approximate an eigenpair $(\lambda,\vec{q})$ of $A$}
%
    \For {$i = 1,2,\dots,k$}
    {
      $\vec{z_k} \gets A \vec{q_{k-1}}$ \;
      $\vec{q_k} \gets \dfrac{\vec{z_k}}{\norm{\vec{z_k}}_2}$ \;
      $\lambda_k \gets \trans{\vec{q_k}} A \vec{q_k}$ \;
    }
%
    return $(\lambda_k, \vec{q_k})$ \;
\end{algorithm}
\hfill

One immediate trick that is detected, is that we are not giving a
precise value for $k$, the number of iterations; this is because we do
not really know how many in advance, though we know how ``fast'' we
can reach convergence, more about this in a minute. Though the
algorithm looks trivial, a powerful theorem justifies why it
works. Golub mentions the  conditions which are required for its
convergence: the 
maximum eigenvalue of $A$ must be unique (no repetition), and the 
initial vector $q_0$ is not ``deficient''  (its component on the
direction of the eigenvector with maximum value must not be zero). A
proof of convergence/correctness can be consulted in \cite{golub13}. \\\

Golub also mentions the computable error bounds of this method.
The real eigenvalue and eigenvectors of $A$ will satisfy the equation
below: 

\[
A \vec{q} = \lambda \vec{q}
\]
\hfill

But accepting the fact that a computer will not product exactly the
eigenvalue nor the eigenvector, we can at least see how close we are
in meeting above condition. That is, we can calculate the error $\delta$: 

\[
\norm{A \vec{q_k} - \lambda_k \vec{q_k}}_2 = \delta
\]
\hfill

Golub shows that there is an eigenvalue $\lambda$ that satisfies
$\abs{\lambda_k - \lambda} \le \sqrt{2}\delta$; which is a way to tell
that we can really approximate an actual eigenvalue, as long as we are
capable of reproducing its defining property with good accuracy (which
in turn, will depend on how many iterations we make). \\

Alright, so we know how to calculate one eigenpair; why not
calculating them all? We may be tempted now to recall the geometric
proof of SVD (see \cref{cha:svd-theory}), and consider the following
procedure for finding all the eigenpairs (assuming preconditions met):

\begin{enumerate}
\item Pick carefully initial vector. 
\item Apply \cref{alg:power-method} to find the first
  eigenpair. 
\item Obtain the hyperplane that is orthogonal to the first
  eigenvector found, and repeat recursively the procedure until
  we have all the desired eigenpairs\footnote{The third step is
    usually called ``deflation'' (see \cite{golub13}), when mentioned
    in the context of the matrix, as it is reduced to dimensions $(n-1)
    \times (n-1)$.}.
\end{enumerate}
\hfill

The problem with this procedure, also exposed by Golub in his proof of
correctness, is that the rate of convergence depends on
$\abs{\frac{\lambda_2}{\lambda_1}}^k$; where $\lambda_2$ is the second
largest eigenvalue in absolute value. Thus, unless there is a
considerable gap between first and second largest eigenvalues of $A$,
the Power Method will converge quite slowly. That makes it unsuitable
for practical purposes, at least in the standalone version we just
presented. Further sections will show how it can evolve to overcome
this limitation. 
